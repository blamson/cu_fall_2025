\section*{Homework 3}

\textbf{Problem} Decide for our running example for which combinations of basic variables we get a basic feasible or infeasible solution via a computation. (From lecture 4).

I'll be using the same system of equations as in homework 2.

\[
	A = \m{v}{1&2&3&0 \\ 1&1&1&1}, \;\; b = \m{p}{6\\4}
\]

To get our basic solution we need to take $A$ and take 2 subsets of columns from it, $B$ and $N$. The columns we choose for these don't need to be consecutive or anything. For the sake of this assignment, we'll nab 2 columns from $A$ to use for $B$ and the other two will go to $N$. 

We'll be ignoring $N$ after that as it's set to 0. 

Our basic solution will be calculated as such:

\[
	x_B = B^{-1}b
\]

To choose something different from the lecture, I'll be using columns 2 and 3 for $B$. $N$ then will get columns 1 and 4.

\[
	B = \m{v}{2&3\\1&1}, \;\; N = \m{v}{1&0\\1&1}
\]

From there it is as simple as doing the calculation.

\begin{align}
	x_B &= B^{-1}b \\
	&= \m{v}{-1&3\\1&-2} \m{p}{6\\4} \\
	&= \m{p}{6\\-2} \\
\end{align}

So this is giving us the basic solution of $x_2 = 6, x_3 = -2$. Because $-2 < 0$, this contradicts the non-negativity constraint for $x_3$. Thus, this represents an infeasible solution for the system.
