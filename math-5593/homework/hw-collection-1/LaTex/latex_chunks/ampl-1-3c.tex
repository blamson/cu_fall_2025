\begin{minted}[fontsize=\small,breaklines,linenos]{python}
from amplpy import AMPL
from loguru import logger
import polars as pl
import plotly.express as px

def main(reheat_hours_list):
    file_name = "steel4"
    profit_list = []
    duals = []

    # Goal here is to populate a dataframe we can examine through all of the necessary runs
    # We initiate AMLP() in this loop so thats its fully reset between runs.
    # Not doing this can result in weird results for edge case scenarios.
    for reheat_hours in reheat_hours_list:
        logger.info("Initializing solver")
        ampl = AMPL()
        ampl.setOption("solver", "highs")
        logger.info("Reading data")
        ampl.read(f"../models/{file_name}.mod")
        ampl.read_data(f"../data/{file_name}.dat")
        logger.info(f"Solving with reheat availability = {reheat_hours}")
        ampl.getParameter("avail")["reheat"] = reheat_hours
        ampl.solve()

        profit = ampl.getObjective("Total_Profit").value()
        profit_list.append(round(profit, 2))

        # Dual value (shadow price) for the reheat stage
        dual_value = ampl.getConstraint("Time")["reheat"].dual()
        duals.append(round(dual_value, 2))

    df = pl.DataFrame({
        "reheat_hours": reheat_hours_list,
        "profit": profit_list,
        "time_dual_value": duals
    }, strict=False)


    return df


if __name__ == "__main__":
    reheat_hours_list = list(range(10, 41))
    # test_value_1 = 37 + (9/14)
    # test_value_2 = 37 + (10/14)
    # reheat_hours_list = [test_value_1 - 1, test_value_1, test_value_2 - 1, test_value_2]
    # reheat_hours_list = [test_value_1, test_value_2]

    df = main(reheat_hours_list)
    with pl.Config(tbl_rows=40):
        print(df)
    df.write_csv("../data/reheat-profit-table.csv")

    # fig = px.line(df, x="reheat_hours", y="profit, title='Profit vs. Reheat Hours')
    # fig.show()
\end{minted}
