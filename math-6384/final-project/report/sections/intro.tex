\section*{Introduction}

\subsection*{Background Info}

The topic of bee populations in the United States is not a new one. We have been aware of the dwindling bee population in this country for decades. Though this topic has a lot of surface level popular appeal through the "save the bees" movement, there is a lot that still isn't understood about this trend. For starters, it's difficult to pin down exact numbers on this decline as it depends on location, species and many other factors. There are also many known overlapping challenges facing native wild bee populations and non-native human managed honeybee colonies. Native wild bee populations have been harmed by habitat loss, urbanization, pesticide use, and climate change to name a few. Honeybee colonies have been ravaged by parasites, disease, other non-native bugs such as the small hive beetle and a bizarre phenomenon called Colony Collapse Disorder (Dr. Underwood, \href{https://extension.psu.edu/a-quick-reference-guide-to-honey-bee-parasites-pests-predators-and-diseases}{source}). This is a complicated and multifaceted issue; there isn't just a single simple cause driving these populations down. 

The importance of this decline goes beyond a simple adoration of the fluffy insects. According to Brianna Randall from the NRCS, "More than 80 percent of the world’s flowering plants need a pollinator to reproduce; and we need pollinators too, since most of our food comes from flowering plants. One out of every three bites of our food, including fruits, vegetables, chocolate, coffee, nuts, and spices, is created with the help of pollinators." (\href{https://www.farmers.gov/blog/value-birds-and-bees}{source}). There are so many managed honeybee colonies in the US because they are crucial for the agricultural branch of our economy to function. As bee populations struggle there are powerful ripple effects that are felt both on the local ecological scale and at the urban level. Referring back to Brianna again, "pollinators' ecological service is valued at \$200 billion each year". Meanwhile native bees are the backbone of every states ecological wellbeing by pollinating native flora and being essential parts of the food chain. 

I am particularly interested in the challenges to habitat that bees are facing. As climate change and urbanization destroy hives and make previous areas infeasible, I am curious to see if this is reflected in spatial data of bees. If we look at maps over time, do the spatial distributions of bees change? Do we see clustering in new areas or an absence of clustering in areas from prior years? Do the spatial scales of clustering change over time and if so, what does that mean? 

\subsection*{Data Source}

The data I am using for this project comes from the Global Biodiversity Information Facility (GBIF). This is an international group funded by the world's governments that has the explicit goal of "providing anyone, anywhere, open access to data about all types of life on Earth" (\href{https://www.gbif.org/what-is-gbif}{source}). The dataset I'm working with specifically is a collection of insect occurrence records across all of the United States of America, and some areas outside of it, with a focus on bees 
(\href{https://www.gbif.org/dataset/f519367d-6b9d-411c-b319-99424741e7de}{source}). 

This dataset is an aggregation of many different projects and collection efforts across various groups. These groups include USGS employees, federal workers, volunteers, private groups and civilian scientists. It contains an enormous amount of species occurrence records for "native and non-native bees, wasps and other insects". These data were collected using a variety of methods such as pan, malaise and vane trapping. This dataset features occurrences going all the way back to 1996 complete with latitude and longitude information, so I'll be treating this as point process data and examining trends across decades. 