\section*{Conclusion}

For overall conclusions, there does seem to be evidence of a change in spatial patterns of bee occurrences in Maryland across recent decades. There appears to have been a push somewhere in the 2010s to the northern parts of the state which could either be a change in bee behavior or a shift in data collection practices. The 2020s show particularly interesting patterns with what appears to be a consolidation of occurrences in two key areas. The reasons for this shift are unclear, but there appears to be strong evidence to indicate that something has changed since the 2000s. 

Conclusions from this project must be made with a degree of caution. There are a few key limitations that I feel were not adequately addressed. The first limitation is that this analysis includes both native and non-native bees. The non-native honeybee is a controlled commercial bee managed by human bee-keepers. As such, there is a concern that spatial patterns in the data may be influenced by artifical patterns of bee-keepers and the locations of their hives. It is possible that an analysis that looked at only "domesticated" bees or wild bees would provide very different results as they have different factors influencing their behavior. 

Second, there is a level of concern at the amount of overlapping points in the Maryland dataset. With 190 thousand rows and only 2747 unique locations, the jittering solution leveraged in this project may have been very inappropriate. It is possible that some sort of counting aggregation per point or a shift to a regional perspective would have been better for the analysis done here. Points in the original dataset that fell outside of the bounds of Maryland while being labeled as Maryland also call into question the overall reliability of the precision of the location data used. 

As well, the chosen sample size for this project may have been problematic. Using only 1500 out of 190000 rows introduces an underlying concern that patterns shown in this analysis may not be truly representative of overall trends. For the sake of this project and its scope this stratified sampling was necessary, but all interpretations must be made with that in mind. It's hard to know if a different random seed would have chosen very different results. 

Lastly, and perhaps most importantly, is a lack of rigorous understanding of the data collection process for this project. When examining data like this, it is crucial that one is able to identify what patterns represent genuine bee behavior and what patterns represent shifts in the data collection process. That is not an understanding present here, thus it is very likely that any results shown in this project are more representative of shifts in the process of bee occurrence research in Maryland than any true shifts in the spatial patterns of bees. 

I do believe the analysis shown here could lead into far more interesting research downstream though. Once a better understanding of how data collection was performed for this dataset, it could be worthwhile to compare the spatial patterns of bees native to Maryland to those that are not. Performing this analysis over time could potentially show enlightening results such as non-native bees moving into native bee areas. A deeper focus on native bees would also be interesting in general as those are the types of bees more effected by changes in their habitats and their population cannot be artifically inflated by an import of more bees. Spatial patterns of native bees are likely more organic than those of honeybees as well. 

Overall this project was a very interesting one to work on. Learning to handle different coordinate reference systems and handling overlapping points bring up many questions on what it really means to work with real world spatial data. A lot of care is required to handle it, and this also extends to the interpretation of results. Case control tests with multiple categories are also interesting, and I am curious how this kind of data would be handled by one more experienced than myself. The more I read on the topic of bees the more I realized how deep and nuanced this kind of topic is. The commercialization of honeybees and the varied nesting methods of native bees like sweat bees that nest in the soil or in rotting wood are two good examples. This is all stuff that goes so much deeper than plotting some points on a map and I have a far deeper appreciation for the people who dedicate their lives to this kind of work.